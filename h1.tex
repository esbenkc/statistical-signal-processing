% Options for packages loaded elsewhere
\PassOptionsToPackage{unicode}{hyperref}
\PassOptionsToPackage{hyphens}{url}
%
\documentclass[
]{article}
\title{Homework 1}
\author{Esben Kran}
\date{4/6/2022}

\usepackage{amsmath,amssymb}
\usepackage{lmodern}
\usepackage{iftex}
\ifPDFTeX
  \usepackage[T1]{fontenc}
  \usepackage[utf8]{inputenc}
  \usepackage{textcomp} % provide euro and other symbols
\else % if luatex or xetex
  \usepackage{unicode-math}
  \defaultfontfeatures{Scale=MatchLowercase}
  \defaultfontfeatures[\rmfamily]{Ligatures=TeX,Scale=1}
\fi
% Use upquote if available, for straight quotes in verbatim environments
\IfFileExists{upquote.sty}{\usepackage{upquote}}{}
\IfFileExists{microtype.sty}{% use microtype if available
  \usepackage[]{microtype}
  \UseMicrotypeSet[protrusion]{basicmath} % disable protrusion for tt fonts
}{}
\makeatletter
\@ifundefined{KOMAClassName}{% if non-KOMA class
  \IfFileExists{parskip.sty}{%
    \usepackage{parskip}
  }{% else
    \setlength{\parindent}{0pt}
    \setlength{\parskip}{6pt plus 2pt minus 1pt}}
}{% if KOMA class
  \KOMAoptions{parskip=half}}
\makeatother
\usepackage{xcolor}
\IfFileExists{xurl.sty}{\usepackage{xurl}}{} % add URL line breaks if available
\IfFileExists{bookmark.sty}{\usepackage{bookmark}}{\usepackage{hyperref}}
\hypersetup{
  pdftitle={Homework 1},
  pdfauthor={Esben Kran},
  hidelinks,
  pdfcreator={LaTeX via pandoc}}
\urlstyle{same} % disable monospaced font for URLs
\usepackage[margin=1in]{geometry}
\usepackage{color}
\usepackage{fancyvrb}
\newcommand{\VerbBar}{|}
\newcommand{\VERB}{\Verb[commandchars=\\\{\}]}
\DefineVerbatimEnvironment{Highlighting}{Verbatim}{commandchars=\\\{\}}
% Add ',fontsize=\small' for more characters per line
\usepackage{framed}
\definecolor{shadecolor}{RGB}{248,248,248}
\newenvironment{Shaded}{\begin{snugshade}}{\end{snugshade}}
\newcommand{\AlertTok}[1]{\textcolor[rgb]{0.94,0.16,0.16}{#1}}
\newcommand{\AnnotationTok}[1]{\textcolor[rgb]{0.56,0.35,0.01}{\textbf{\textit{#1}}}}
\newcommand{\AttributeTok}[1]{\textcolor[rgb]{0.77,0.63,0.00}{#1}}
\newcommand{\BaseNTok}[1]{\textcolor[rgb]{0.00,0.00,0.81}{#1}}
\newcommand{\BuiltInTok}[1]{#1}
\newcommand{\CharTok}[1]{\textcolor[rgb]{0.31,0.60,0.02}{#1}}
\newcommand{\CommentTok}[1]{\textcolor[rgb]{0.56,0.35,0.01}{\textit{#1}}}
\newcommand{\CommentVarTok}[1]{\textcolor[rgb]{0.56,0.35,0.01}{\textbf{\textit{#1}}}}
\newcommand{\ConstantTok}[1]{\textcolor[rgb]{0.00,0.00,0.00}{#1}}
\newcommand{\ControlFlowTok}[1]{\textcolor[rgb]{0.13,0.29,0.53}{\textbf{#1}}}
\newcommand{\DataTypeTok}[1]{\textcolor[rgb]{0.13,0.29,0.53}{#1}}
\newcommand{\DecValTok}[1]{\textcolor[rgb]{0.00,0.00,0.81}{#1}}
\newcommand{\DocumentationTok}[1]{\textcolor[rgb]{0.56,0.35,0.01}{\textbf{\textit{#1}}}}
\newcommand{\ErrorTok}[1]{\textcolor[rgb]{0.64,0.00,0.00}{\textbf{#1}}}
\newcommand{\ExtensionTok}[1]{#1}
\newcommand{\FloatTok}[1]{\textcolor[rgb]{0.00,0.00,0.81}{#1}}
\newcommand{\FunctionTok}[1]{\textcolor[rgb]{0.00,0.00,0.00}{#1}}
\newcommand{\ImportTok}[1]{#1}
\newcommand{\InformationTok}[1]{\textcolor[rgb]{0.56,0.35,0.01}{\textbf{\textit{#1}}}}
\newcommand{\KeywordTok}[1]{\textcolor[rgb]{0.13,0.29,0.53}{\textbf{#1}}}
\newcommand{\NormalTok}[1]{#1}
\newcommand{\OperatorTok}[1]{\textcolor[rgb]{0.81,0.36,0.00}{\textbf{#1}}}
\newcommand{\OtherTok}[1]{\textcolor[rgb]{0.56,0.35,0.01}{#1}}
\newcommand{\PreprocessorTok}[1]{\textcolor[rgb]{0.56,0.35,0.01}{\textit{#1}}}
\newcommand{\RegionMarkerTok}[1]{#1}
\newcommand{\SpecialCharTok}[1]{\textcolor[rgb]{0.00,0.00,0.00}{#1}}
\newcommand{\SpecialStringTok}[1]{\textcolor[rgb]{0.31,0.60,0.02}{#1}}
\newcommand{\StringTok}[1]{\textcolor[rgb]{0.31,0.60,0.02}{#1}}
\newcommand{\VariableTok}[1]{\textcolor[rgb]{0.00,0.00,0.00}{#1}}
\newcommand{\VerbatimStringTok}[1]{\textcolor[rgb]{0.31,0.60,0.02}{#1}}
\newcommand{\WarningTok}[1]{\textcolor[rgb]{0.56,0.35,0.01}{\textbf{\textit{#1}}}}
\usepackage{graphicx}
\makeatletter
\def\maxwidth{\ifdim\Gin@nat@width>\linewidth\linewidth\else\Gin@nat@width\fi}
\def\maxheight{\ifdim\Gin@nat@height>\textheight\textheight\else\Gin@nat@height\fi}
\makeatother
% Scale images if necessary, so that they will not overflow the page
% margins by default, and it is still possible to overwrite the defaults
% using explicit options in \includegraphics[width, height, ...]{}
\setkeys{Gin}{width=\maxwidth,height=\maxheight,keepaspectratio}
% Set default figure placement to htbp
\makeatletter
\def\fps@figure{htbp}
\makeatother
\setlength{\emergencystretch}{3em} % prevent overfull lines
\providecommand{\tightlist}{%
  \setlength{\itemsep}{0pt}\setlength{\parskip}{0pt}}
\setcounter{secnumdepth}{-\maxdimen} % remove section numbering
\ifLuaTeX
  \usepackage{selnolig}  % disable illegal ligatures
\fi

\begin{document}
\maketitle

\hypertarget{statistical-signal-processing}{%
\section{Statistical signal
processing}\label{statistical-signal-processing}}

\begin{Shaded}
\begin{Highlighting}[]
\NormalTok{pacman}\SpecialCharTok{::}\FunctionTok{p\_load}\NormalTok{(tidyverse, ggplot)}
\end{Highlighting}
\end{Shaded}

\begin{verbatim}
## Installing package into 'C:/Users/esben/Documents/R/win-library/4.0'
## (as 'lib' is unspecified)
\end{verbatim}

\begin{verbatim}
## Warning: package 'ggplot' is not available (for R version 4.0.2)
\end{verbatim}

\begin{verbatim}
## Warning: unable to access index for repository http://www.stats.ox.ac.uk/pub/RWin/bin/windows/contrib/4.0:
##   cannot open URL 'http://www.stats.ox.ac.uk/pub/RWin/bin/windows/contrib/4.0/PACKAGES'
\end{verbatim}

\begin{verbatim}
## Warning in p_install(package, character.only = TRUE, ...):
\end{verbatim}

\begin{verbatim}
## Warning in library(package, lib.loc = lib.loc, character.only = TRUE,
## logical.return = TRUE, : there is no package called 'ggplot'
\end{verbatim}

\begin{verbatim}
## Warning in pacman::p_load(tidyverse, ggplot): Failed to install/load:
## ggplot
\end{verbatim}

\hypertarget{section}{%
\subsection{2.2.1}\label{section}}

{[}NA{]}

\hypertarget{section-1}{%
\subsection{1.1.1}\label{section-1}}

In a radar system an estimator of round trip delay \(\tau_0\) has the
PDF \(\hat \tau_0 \sim N(\tau_0,\sigma_{\hat\tau_0}^2)\), where
\(\tau_0\) is the true value. If the range is to be estimated, propose
an estimator \(R\) and find its PDF. Next determine the standard
deviation \(\sigma_{\hat\tau_0}\) so that 99\% of the time the range
estimate will be within 100 m of the true value. Use
\(c=3\cdot 10^8 m/s\) for the speed of electromagnetic propagation.

The estimator for R is
\[R~\dfrac{c\cdot N(\tau_0, \sigma^2_{\hat \tau_0})}{2}\]

\begin{Shaded}
\begin{Highlighting}[]
\NormalTok{c }\OtherTok{\textless{}{-}} \DecValTok{3} \SpecialCharTok{*} \DecValTok{10}\SpecialCharTok{\^{}}\DecValTok{8}
\NormalTok{range\_lim }\OtherTok{=} \DecValTok{100}
\NormalTok{sigma }\OtherTok{=} \DecValTok{30}
\NormalTok{tau }\OtherTok{=} \DecValTok{200}

\CommentTok{\# Defining range estimator:}
\NormalTok{R }\OtherTok{\textless{}{-}} \ControlFlowTok{function}\NormalTok{(x) (c}\SpecialCharTok{*}\FunctionTok{dnorm}\NormalTok{(x, tau, sigma)) }\SpecialCharTok{/} \DecValTok{2}
\FunctionTok{plot}\NormalTok{(R, }\AttributeTok{xlim=}\FunctionTok{c}\NormalTok{(}\DecValTok{1}\NormalTok{,}\DecValTok{400}\NormalTok{))}
\end{Highlighting}
\end{Shaded}

\includegraphics{h1_files/figure-latex/unnamed-chunk-2-1.pdf}

\begin{Shaded}
\begin{Highlighting}[]
\CommentTok{\# Not sure about determining SD for 99\% range estimate \textbackslash{}in [{-}100, 100]}
\end{Highlighting}
\end{Shaded}

\hypertarget{section-2}{%
\subsection{1.1.4}\label{section-2}}

It is desired to estimate the value of a DC level A in WGN or
\[x[n]=A+w[n]$ where $n=0,1,...,N-1\] Where \(w[n]\) is zero mean and
uncorrelated, and each sample has variance \(\sigma^2=1\). Consider the
two estimators: \[\hat A = \dfrac{1}{N}\sum^{N-1}_{n=0}{x[n]}\]
\[\overline A=\dfrac{1}{N+2}\left(2x[0]+\sum^{N-1}_{n=1}{x[n]+2x[N-1]}\right)\]
Which one is better? Does it depend on the value of A?

Run M experiments to see difference between the two estimators and
analyse error. Based on what we can see here, \textbf{the estimator
\(A_1\) is always better than estimator \(A_2\)}.

\begin{Shaded}
\begin{Highlighting}[]
\NormalTok{test\_range }\OtherTok{=} \FunctionTok{seq}\NormalTok{(}\SpecialCharTok{{-}}\DecValTok{10}\NormalTok{, }\DecValTok{10}\NormalTok{, }\FloatTok{0.1}\NormalTok{)}
\NormalTok{samples }\OtherTok{=} \DecValTok{100}
\NormalTok{M }\OtherTok{=} \DecValTok{1000}

\NormalTok{a1 }\OtherTok{\textless{}{-}} \ControlFlowTok{function}\NormalTok{(x) }\FunctionTok{mean}\NormalTok{(x)}
\NormalTok{a2 }\OtherTok{\textless{}{-}} \ControlFlowTok{function}\NormalTok{(x) (}\DecValTok{1}\SpecialCharTok{/}\NormalTok{(}\FunctionTok{length}\NormalTok{(x)}\SpecialCharTok{+}\DecValTok{2}\NormalTok{))}\SpecialCharTok{*}\NormalTok{(}\DecValTok{2}\SpecialCharTok{*}\NormalTok{x[}\DecValTok{1}\NormalTok{]}\SpecialCharTok{+}\FunctionTok{sum}\NormalTok{(x)}\SpecialCharTok{+}\DecValTok{2}\SpecialCharTok{*}\NormalTok{x[}\FunctionTok{length}\NormalTok{(x)}\SpecialCharTok{{-}}\DecValTok{1}\NormalTok{])}

\FunctionTok{data.frame}\NormalTok{(}
    \AttributeTok{m =} \FunctionTok{lapply}\NormalTok{(}\DecValTok{1}\SpecialCharTok{:}\NormalTok{M, }\ControlFlowTok{function}\NormalTok{(m) }\FunctionTok{rep}\NormalTok{(m, }\FunctionTok{length}\NormalTok{(test\_range) }\SpecialCharTok{*}\NormalTok{ samples)) }\SpecialCharTok{\%\textgreater{}\%}\NormalTok{ unlist,}
    \AttributeTok{mu =} \FunctionTok{lapply}\NormalTok{(}\DecValTok{1}\SpecialCharTok{:}\NormalTok{M, }\ControlFlowTok{function}\NormalTok{(m) }\FunctionTok{lapply}\NormalTok{(test\_range, }\ControlFlowTok{function}\NormalTok{(i) }\FunctionTok{rep}\NormalTok{(i, samples)) }\SpecialCharTok{\%\textgreater{}\%}\NormalTok{ unlist) }\SpecialCharTok{\%\textgreater{}\%}\NormalTok{ unlist,}
    \AttributeTok{x =} \FunctionTok{lapply}\NormalTok{(}\DecValTok{1}\SpecialCharTok{:}\NormalTok{M, }\ControlFlowTok{function}\NormalTok{(m) }\FunctionTok{lapply}\NormalTok{(test\_range, }\ControlFlowTok{function}\NormalTok{(i) }\FunctionTok{rnorm}\NormalTok{(samples, i, }\DecValTok{1}\NormalTok{)) }\SpecialCharTok{\%\textgreater{}\%}\NormalTok{ unlist) }\SpecialCharTok{\%\textgreater{}\%}\NormalTok{ unlist}
\NormalTok{  ) }\SpecialCharTok{\%\textgreater{}\%} 
  \FunctionTok{group\_by}\NormalTok{(mu, m) }\SpecialCharTok{\%\textgreater{}\%} 
  \FunctionTok{summarise}\NormalTok{(}
    \AttributeTok{a1 =} \FunctionTok{abs}\NormalTok{(}\FunctionTok{a1}\NormalTok{(x)}\SpecialCharTok{{-}}\NormalTok{mu),}
    \AttributeTok{a2 =} \FunctionTok{abs}\NormalTok{(}\FunctionTok{a2}\NormalTok{(x)}\SpecialCharTok{{-}}\NormalTok{mu)}
\NormalTok{  ) }\SpecialCharTok{\%\textgreater{}\%} 
  \FunctionTok{group\_by}\NormalTok{(mu) }\SpecialCharTok{\%\textgreater{}\%} 
  \FunctionTok{summarise}\NormalTok{(}
    \AttributeTok{a1 =} \FunctionTok{mean}\NormalTok{(a1),}
    \AttributeTok{a2 =} \FunctionTok{mean}\NormalTok{(a2)}
\NormalTok{  ) }\SpecialCharTok{\%\textgreater{}\%} 
  \FunctionTok{unique}\NormalTok{() }\SpecialCharTok{\%\textgreater{}\%} 
  \FunctionTok{pivot\_longer}\NormalTok{(}\FunctionTok{c}\NormalTok{(a1, a2)) }\SpecialCharTok{\%\textgreater{}\%} 
  \FunctionTok{ggplot}\NormalTok{() }\SpecialCharTok{+}
  \FunctionTok{aes}\NormalTok{(mu, value, }\AttributeTok{color =}\NormalTok{ name) }\SpecialCharTok{+}
  \FunctionTok{geom\_line}\NormalTok{() }\SpecialCharTok{+}
  \FunctionTok{theme\_classic}\NormalTok{() }\SpecialCharTok{+}
  \FunctionTok{labs}\NormalTok{(}
    \AttributeTok{x =} \StringTok{"Tau"}\NormalTok{,}
    \AttributeTok{y =} \StringTok{"Error"}\NormalTok{,}
    \AttributeTok{color =} \StringTok{"Estimator"}
\NormalTok{  ) }\SpecialCharTok{+}
  \FunctionTok{coord\_cartesian}\NormalTok{(}\AttributeTok{ylim=}\FunctionTok{c}\NormalTok{(}\DecValTok{0}\NormalTok{, }\FloatTok{0.2}\NormalTok{), }\AttributeTok{expand=}\NormalTok{F)}
\end{Highlighting}
\end{Shaded}

\begin{verbatim}
## `summarise()` has grouped output by 'mu', 'm'. You can override using the `.groups` argument.
\end{verbatim}

\includegraphics{h1_files/figure-latex/unnamed-chunk-3-1.pdf}

\hypertarget{section-3}{%
\subsection{2.1.1}\label{section-3}}

{[}NA{]}

\hypertarget{section-4}{%
\subsection{1.1.2}\label{section-4}}

An unknown parameter \(O\) influences the outcome of an experiment which
is modeled by the random variable x. The PDF of x is
\[p(x;O)=\dfrac{1}{\sqrt{2\pi}}exp\left[-\dfrac{1}{2}(x-O)^2\right]\] A
series of experiments is performed, and x is found to always be in the
interval \([97,103]\). As a result, the investigator concludes that
\(O\) must have been 100. Is this assertion correct?

\begin{Shaded}
\begin{Highlighting}[]
\NormalTok{a }\OtherTok{\textless{}{-}} \DecValTok{100}
\NormalTok{x }\OtherTok{\textless{}{-}} \FunctionTok{rnorm}\NormalTok{(}\DecValTok{1000}\NormalTok{, }\DecValTok{100}\NormalTok{, }\DecValTok{1}\NormalTok{)}
\NormalTok{mean\_x }\OtherTok{\textless{}{-}} \FunctionTok{mean}\NormalTok{(x)}
\NormalTok{range\_x }\OtherTok{\textless{}{-}} \FunctionTok{max}\NormalTok{(x) }\SpecialCharTok{{-}} \FunctionTok{min}\NormalTok{(x)}

\FunctionTok{paste}\NormalTok{(}
  \StringTok{"**[1.1.2] Sampling a 1000 times with an A of"}\NormalTok{,}
\NormalTok{  a,}
  \StringTok{"gives a mean of"}\NormalTok{,}
\NormalTok{  mean\_x,}
  \StringTok{"with range"}\NormalTok{,}
  \FunctionTok{min}\NormalTok{(x),}
  \StringTok{"to"}\NormalTok{,}
  \FunctionTok{max}\NormalTok{(x),}
  \StringTok{"of"}\NormalTok{,}
\NormalTok{  range\_x,}
  \StringTok{"which indicates that the"}\NormalTok{,}
  \StringTok{"investigator is correct.**"}
\NormalTok{)}
\end{Highlighting}
\end{Shaded}

\begin{verbatim}
## [1] "**[1.1.2] Sampling a 1000 times with an A of 100 gives a mean of 99.9723798909143 with range 96.6226832176055 to 103.283011297738 of 6.66032808013273 which indicates that the investigator is correct.**"
\end{verbatim}

\hypertarget{section-5}{%
\subsection{6}\label{section-5}}

Suppose \(x\sim N(5,2)\) and \(y\sim 2x+4\). Find \(E(y)\), \(var(y)\),
and the PDF \(p_y(y)\).

Expected value (mean(y)):

\begin{Shaded}
\begin{Highlighting}[]
\NormalTok{beta\_1 }\OtherTok{\textless{}{-}} \ControlFlowTok{function}\NormalTok{(x)}
  \FunctionTok{dnorm}\NormalTok{(x, }\DecValTok{5}\NormalTok{, }\DecValTok{2}\NormalTok{)}
\NormalTok{f }\OtherTok{\textless{}{-}} \ControlFlowTok{function}\NormalTok{(x)}
  \DecValTok{1}\SpecialCharTok{/}\FunctionTok{abs}\NormalTok{(}\DecValTok{2}\NormalTok{) }\SpecialCharTok{*} \FunctionTok{beta\_1}\NormalTok{((x}\DecValTok{{-}4}\NormalTok{)}\SpecialCharTok{/}\DecValTok{2}\NormalTok{)}

\FunctionTok{weighted.mean}\NormalTok{(}\FunctionTok{seq}\NormalTok{(}\SpecialCharTok{{-}}\FloatTok{1e2}\NormalTok{, }\FloatTok{1e2}\NormalTok{, }\FloatTok{1e{-}2}\NormalTok{), }\FunctionTok{f}\NormalTok{(}\FunctionTok{seq}\NormalTok{(}\SpecialCharTok{{-}}\FloatTok{1e2}\NormalTok{, }\FloatTok{1e2}\NormalTok{, }\FloatTok{1e{-}2}\NormalTok{)))}
\end{Highlighting}
\end{Shaded}

\begin{verbatim}
## [1] 14
\end{verbatim}

Variance from the sampled distribution:

\begin{Shaded}
\begin{Highlighting}[]
\NormalTok{x }\OtherTok{\textless{}{-}} \FunctionTok{seq}\NormalTok{(}\DecValTok{0}\NormalTok{, }\DecValTok{50}\NormalTok{, }\FloatTok{0.1}\NormalTok{)}
\NormalTok{px }\OtherTok{\textless{}{-}} \FunctionTok{f}\NormalTok{(}\FunctionTok{seq}\NormalTok{(}\DecValTok{0}\NormalTok{, }\DecValTok{50}\NormalTok{, }\FloatTok{0.1}\NormalTok{))}
\NormalTok{draws }\OtherTok{\textless{}{-}} \FunctionTok{sample}\NormalTok{(x, }\AttributeTok{size =} \DecValTok{5000}\NormalTok{, }\AttributeTok{replace =} \ConstantTok{TRUE}\NormalTok{, }\AttributeTok{prob =}\NormalTok{ px)}

\FunctionTok{var}\NormalTok{(draws)}
\end{Highlighting}
\end{Shaded}

\begin{verbatim}
## [1] 15.6968
\end{verbatim}

The PDF of y with \(a\) multiplied into \(beta_1\):

\begin{Shaded}
\begin{Highlighting}[]
\FunctionTok{plot}\NormalTok{(f, }\AttributeTok{ylim=}\FunctionTok{c}\NormalTok{(}\DecValTok{0}\NormalTok{,}\FloatTok{0.2}\NormalTok{), }\AttributeTok{xlim=}\FunctionTok{c}\NormalTok{(}\DecValTok{0}\NormalTok{, }\DecValTok{30}\NormalTok{))}
\end{Highlighting}
\end{Shaded}

\includegraphics{h1_files/figure-latex/unnamed-chunk-7-1.pdf}

\hypertarget{section-6}{%
\subsection{7}\label{section-6}}

Suppose that \(x\sim N(0,\sigma_x^2)\) and \(w\sim N(0,\sigma_w^2)\) and
\(y=ax+w\). If w and x are independent, what is mean and covariance
matrix for the Gaussian vector \(z=[x,y]^T\)? Hint: Note that \(E[z]\)
should be a 2D column vector and \(var(z)\) should be a 2x2 matrix.

\[z=[x,y]^T\] \[[E(x), E(\omega)]^T  = [0, 0]^T\]
\[var(z)= \begin{bmatrix}\sigma_x^2&cov(x,y)\\cov(x,y)&var(y)\end{bmatrix}\]

\end{document}
